% !TEX root = QlockToo.tex

\section{Konstruktion und Produktion}
\label{sec:KonstruktionFertigung}

\begin{multicols}{2}
Um Kosten zu sparen wird die ClockToo als eine Kombination aus Kaufteilen und eigener Produktion ausgeführt. Ziel ist eine möglichst getreue Nachbildung der Original CLOCKTWO® der Biegert~\&~Funk Manufacture GmbH \& Co. KG. 

\subsection{Das Uhrgehäuse}

Die Basis für das Uhrengehäuse bildet der IKEA-Bilderrahmen RIBBA in schwarz. Die Bilddarstellungsmaße von 50 x 50 cm eigenen sich perfekt, um die Buchstabenmatrix zur Geltung zu bringen, sodass die Wortuhr die gewünschte Wirkung erzielt. In dem besonders tiefen Rahmen,  Tiefe~=~4,5~cm, kann die komplette LED-Matrix inklusive der Elektronik untergebracht werden. Lediglich zum Hinausführen des Stromkabels und zur Anbringung der Taster muss der Rahmen modifiziert werden. 

\subsection{LED-Matrix}

Die LED-Matrix wird aus einer quadratischen Spanplatte gefertigt. Diese wird entsprechend der vorgegebenen Buchstabenmatrix 10~x~11 Buchstaben gebohrt und gesenkt, sodass die LEDs die kompletten Buchstaben ausleuchten können. 

\subsection{Buchstabenfolie}

Zur Darstellung der Buchstaben wird eine schwarze Buchstabenfolie, welche die Negativ-Buchstabenmatrix anzeigt, auf eine Plexiglasscheibe geklebt. Um eine möglichst breite Streuung des LED-Lichtes zu erzielen, wird eine zweite Diffusorfolie hinter die Buchstaben geklebt. Trotz der doppelten Folie sieht man die LEDs deutlich hinter den Buchstaben leuchten. Eine vollständige gleichmäßige Ausleuchtung ist nicht möglich. 

Um die Streuung der LEDs zu erhöhen, wird ein zusätzliches Streuplättchen (D:~25~mm) aus transparentem Papier in die gesenkten Kegel vor jede LED mit Flüssigkleber geklebt. 



\end{multicols}

