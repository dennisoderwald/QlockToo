% !TEX root = QlockToo.tex
% Kapitelvorlage

\section{Elektronik}
\label{sec:Elektronik}
Die LED-Matrix mit 10 x 11 Pixel wird von einem Mikrocontroller angesteuert, dies geschieht zeilenweise im Multiplexing-Verfahren. Hierbei wird eine der 10  Zeilen von einem Transistor mit Spannung versorgt und die 11 Pixel der Zeile über einen LED-Treiber simultan angesteuert. Nach einer Millisekunde wird die Zeile ausgeschaltet, neue Daten in den LED-Treiber geschrieben und die nächste Zeile eingeschaltet. 
\subsection{LED-Treiber TLC59116}
Der TLC59116 von Texas Instruments hat 16 PWM-Ausgänge mit 8Bit Auflösung - 255 Helligkeitsstufen - und Stromregelung, die es ermöglicht auf Vorwiderstände an den LED zu verzichten. Der LED-Treiber wird über $I^{2}C$ angesteuert und ist hart mit 0b1100 000[R/W] adressiert. 


\subsection{Arduino Micro}
Anforderungen: serielle Kommunikation, $I^{2}C$-Bus, externer Interrupt, 4x Digital Input 2x Analogeingang, 10x Digital Output, 
Durch regelmäßigen Empfang des DCF77 Zeitsignals ist die Genauigkeit des verbauten Quarz hinreichend genau.