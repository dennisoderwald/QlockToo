% !TEX root = QlockToo.tex
% Kapitelvorlage

\section{Elektronik}
\label{sec:Elektronik}
Die LED-Matrix mit 10 x 11 Pixel wird von einem Mikrocontroller angesteuert, dies geschieht zeilenweise im Multiplexing-Verfahren. Hierbei wird eine der 10  Zeilen von einem Transistor mit Spannung versorgt und die 11 Pixel der Zeile über einen LED-Treiber simultan angesteuert. Nach einer Millisekunde wird die Zeile ausgeschaltet, neue Daten in den LED-Treiber geschrieben und die nächste Zeile eingeschaltet. 
\subsection{LED-Treiber}
Der TLC59116 LED-Treiber von Texas Instruments hat 16 PWM-Ausgänge mit 8Bit Auflösung - 255 Helligkeitsstufen - und eine Stromregelung, die es ermöglicht auf Vorwiderstände an den LED zu verzichten. Der IC wird über $I^{2}C$ angesteuert.Die Adresse kann hardwareseitig in den letzten 4Bit eingestellt werden und ist auf der Platine hart mit 0b1100 000[R/W] adressiert. Der LED-Treiber ist in der SMD TSSOP-28 Bauform.
\subsection{Transistoren}
Die Zeilen werden mit IRF7416 P-FET-Transistoren geschaltet und mit $5V$ versorgt. Die Transistoren werden mit einem Pull-Up Widerstand am Gate beschaltet und mit einem invertierten Signal angesteuert. Die Transistoren haben S0-8 Gehäusebauform. 
\subsection{Taster}
Die vier Kurzhubtaster an der rechten Rahmenseite befinden sich auf einer eigenen Platine mit Verbindungskabel. Zur Entprellung befindet sich kurz vor den vier IO-Pins des Controllers ein RC-Glied. 
\subsection{Sensoren}
Als Helligkeitssensor wird ein LDR und als Temperatursensor ein NTC verwendet. In beiden Fällen mit einem Trimmpoti als Spannungsteiler an einem Analogeingang des Mikrocontrollers.
\subsection{DCF 77}
Das DCF77-Modul empfängt das Mitteleuropäische Zeitsignal und verstärkt es. 
\subsection{Arduino Micro}
Anforderungen: serielle Kommunikation, $I^{2}C$-Bus, externer Interrupt, 4x Digital Input 2x Analogeingang, 10x Digital Output, 
Durch regelmäßigen Empfang des DCF77 Zeitsignals ist die Genauigkeit des verbauten Quarz hinreichend genau.