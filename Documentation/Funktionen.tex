% !TEX root = QlockToo.tex
% Kapitelvorlage

\section{Funktionen}
\label{sec:Funktionen}



Neben der Darstellung der Uhrzeit in Worten verfügt die entwickelte Wortuhr über zahlreiche weitere Funktionen. Während einige Funktionen und Modi bzw. Demonstrationsmuster nur über die Software -~über direkten USB-Anschluss~- ausgewählt werden können, so besteht für die wichtigsten Funktionen die Möglichkeit, diese über vier Taster, welche sich seitlich am Rahmen befinden, auszuwählen. Die Modi, welche über die Software auf die Uhr übertragen werden können, sind im Kapitel~\ref{sec:Software} beschreiben. Die Funktionsbelegung der Taster ist wie folgt:

 %hier Tasterfunktionen MIT Bilder einfügen :) :) :) 
 

\begin{description}

\item[Taster 1 - kurz] Funktion Taster 1
\item[Taster 2]
\item[Taster 3]
\item[Taster 4] Helligkeit - 8 Stufen - aus - Automatik

\end {description}
 
 sieht so noch komisch aus


\textbf{Helligkeit}

Über der Buchstabenmatrix befindet sich ein Helligkeitssensor. Mit Hilfe des Sensors passt sich die Helligkeit der LEDs automatisch an das Umgebungslicht an. So ist die Uhrzeit auch bei Nacht sehr angenehm zu lesen. Alternativ kann die Helligkeit manuell über den Taster vier in acht Stufen eingestellt werden.

